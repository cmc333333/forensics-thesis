%-----------------------------------------------------------------------------
%
%               Template for sigplanconf LaTeX Class
%
% Name:         sigplanconf-template.tex
%
% Purpose:      A template for sigplanconf.cls, which is a LaTeX 2e class
%               file for SIGPLAN conference proceedings.
%
% Guide:        Refer to "Author's Guide to the ACM SIGPLAN Class,"
%               sigplanconf-guide.pdf
%
% Author:       Paul C. Anagnostopoulos
%               Windfall Software
%               978 371-2316
%               paul@windfall.com
%
% Created:      15 February 2005
%
%-----------------------------------------------------------------------------


\documentclass[nocopyrightspace]{sigplanconf}

% The following \documentclass options may be useful:

% preprint      Remove this option only once the paper is in final form.
% 10pt          To set in 10-point type instead of 9-point.
% 11pt          To set in 11-point type instead of 9-point.
% authoryear    To obtain author/year citation style instead of numeric.

\usepackage{amsmath,color,courier,listings,lstcoq,url}
\usepackage[T1]{fontenc}

\definecolor{ltblue}{rgb}{0,0.4,0.4}
\definecolor{dkblue}{rgb}{0,0.1,0.6}
\definecolor{dkgreen}{rgb}{0,0.35,0}
\definecolor{dkviolet}{rgb}{0.3,0,0.5}
\definecolor{dkred}{rgb}{0.5,0,0}

\begin{document}

\lstset{language=coq, basicstyle=\ttfamily\scriptsize, columns=fixed,
breaklines=true}

\special{papersize=8.5in,11in}
\setlength{\pdfpageheight}{\paperheight}
\setlength{\pdfpagewidth}{\paperwidth}

\conferenceinfo{CONF 'yy}{Month d--d, 20yy, City, ST, Country} 
\copyrightyear{20yy} 
\copyrightdata{978-1-nnnn-nnnn-n/yy/mm} 
\doi{nnnnnnn.nnnnnnn}

\titlebanner{banner above paper title}        % These are ignored unless
\preprintfooter{short description of paper}   % 'preprint' option specified.

\title{A Step Towards Formalized Forensics}
\subtitle{Subtitle Text, if any}

\authorinfo{Name1}
           {Affiliation1}
           {Email1}
\authorinfo{Name2\and Name3}
           {Affiliation2/3}
           {Email2/3}

\maketitle

\begin{abstract}
This is the text of the abstract.
\end{abstract}

\category{CR-number}{subcategory}{third-level}

% general terms are not compulsory anymore, 
% you may leave them out
\terms
term1, term2

\keywords
keyword1, keyword2

\section{Introduction}

There is no single definition of essential factors with regard to digital
forensics. Further, the de facto standards are within specific tools. What we
want is a formalization of what it means for a file to be malicious, deleted,
altered, etc. To do that, we use the Coq programming language and proof
environment to formalize a handful of statements along with some
hand-generated proofs for existing file images.

\section{Background}
Ultimately, the long-term goal of this research is to split these two tasks so
that we can share a set of common definitions for what constitutes evidence.
If forensics practitioners could prove that their particular disk image (or
other input) matched agreed-upon definitions, they would have pre-defined
proofs getting their evidence to a meaningful conclusion.

\section{A First Example: File Types}

We start by considering a relatively straight forward request, proving that a
given file is a JPEG. How can we formalize that notion? One tact would be to
use the file name, checking its extension for {\tt jpg} or {\tt jpeg}. We
could also review the JPEG spec and confirm that all meta data is consistent.
We will take a middle route, opting to use ``magic numbers'', that is
tell-tale values within the file data. JPEGs happen to always start with the
bytes {\tt ff d8} and end with {\tt ff d9}, so we will use that to define
evidence that a given file is a JPEG. Similarly, gzipped tar files begin with
{\tt 1f 8b 08} and Linux executables (ELF) begin with {\tt 7f 45 4c 46}.

Writing this in the proof-centric language of Coq (plus some syntactic sugar),
we would see

\begin{lstlisting}
Definition isJpeg (file: File) :=
    file @[  0 ] = value 255
  /\ file @[  1 ] = value 216 
  /\ file @[ -2 ] = value 255
  /\ file @[ -1 ] = value 217.

Definition isTgz (file: File) :=
    file @[ 0 ] = value 31
  /\ file @[ 1 ] = value 139 
  /\ file @[ 2 ] = value 8.

Definition isElf (file: File) :=
    file @[ 0 ] = value 127
  /\ file @[ 1 ] = value 69 
  /\ file @[ 2 ] = value 76
  /\ file @[ 3 ] = value 70.
\end{lstlisting}

That is, the first (and, in the case of JPEGs, last) bytes of a file are both
present (as indicated by {\tt value} and equal to the byte sequences described
above (here, in the form of positive, base-ten numbers. By defining file types
in this manner, we can use them as building blocks within larger proofs. For
example, given these definitions, we can {\it prove} that JPEG files cannot
also be TGZs:

\begin{lstlisting}
Lemma jpeg_is_not_tgz : forall (file: File),
  (isJpeg file) -> ~(isTgz file).
\end{lstlisting}

A full version of this proof (and others mentioned throughout this paper) are
included in the appendix.

\section{Honeynet Example}

Evidence of that style, though not formality, serves as the building blocks
for many forensics cases. Consider next an example \cite{honeynet-15} from the
Honeynet Project's now-defunct {\it Scan of the Month} series. Here, a disk
image involved in a rootkit installation was provided to the security
community. These researchers were challenged to recover a deleted rootkit,
prove that said rootkit was installed, and provide a step-by-step writeup
describing how the rootkit was found.

We use the evidence provided by one of the top entrants, Matt Borland
\cite{borland-honeynet}, as a rough outline of the types of evidence we will
need. This evidence included the description of a deleted, ``tar/gzipped file
containing the tools necessary for creating a home for the attacker on the
compromised system''. More formally, Borland used the fact that a particular
file (designated by Inode \#23) was 1) deleted, 2) a gzipped tar, and 3)
contained malicious files. We will simplify this a bit to come to our
definition in Coq:

\begin{lstlisting}
Definition borland_malicious_tgz (disk: Disk) 
  (inodeIndex: Z): 
  
  exists (tgz_file: File),
    (fileFromInode disk inodeIndex) = value tgz_file
  /\ (isDeleted tgz_file)
  /\ (isTgz tgz_file)
  /\ (tgzContainsMaliciousFiles tgz_file).
\end{lstlisting}

and evidence for this particular competition would complete the proof for

\begin{lstlisting}
Lemma borland_15_proof: 
  (bordland_malicious_tgz honeynet_15_image 23).
\end{lstlisting}

It is important to point out that in Borland's writeup (and in many forensics
papers) the same person is both authoring the {\it definition} of evidence as
well as {\it providing the evidence} itself. This is exactly the problem
discussed above, the sort of biased results we are trying to avoid.

%Note that, while we were given an entire file, we only cared about the initial
%and final bytes. As our definition does not involve any other bytes of the
%image, these bytes could be all zeros, all ones, present, damaged, or
%otherwise. The definition doesn't care, and this is a trait we will use in the
%future to limit the size of our proofs.

\section{Delete INode}

Now, let's consider a more complicated example, that of testing whether a
particular inode index is marked as ``deleted'' in the file system. Note that
there are many possible definitions of ``deleted'' we may choose from. The
ext2 file system has redundant mechanisms to define if a file has been
deleted. We chose one, the allocation bit map. Ultimately, whether or not a
given inode index is marked as deleted involves finding the ``group block''
associated with that index, looking up the bit associated with the index in
the allocation table, and checking whether it is zero (for deleted) or one
(for allocated).

Of course, there is a lot we glossed over. What is a group block? How do we
find the right one? How do we know that the inode index provided is valid? To
answer these questions, we created several additional data structures and
evidence functions.

\section{Future Work}

automatically generate the proof based on disk images; additional types of
evidence; abstractions of file systems


\appendix
\section{Relevant Proofs}

\begin{lstlisting}
Lemma jpeg_is_not_tgz : forall (file: File),
  (isJpeg file) -> ~(isTgz file).
  Proof.
  unfold isTgz, isJpeg.
  intros file jpeg_asmpt.
  destruct jpeg_asmpt as [byte0_is_255].
  rewrite byte0_is_255.
  unfold not. intros contra.
  destruct contra as [not_equal].
  discriminate not_equal.
  Qed.
\end{lstlisting}

\acks

Acknowledgments, if needed.

% We recommend abbrvnat bibliography style.

\bibliographystyle{abbrvnat}

% The bibliography should be embedded for final submission.

\begin{thebibliography}{}
\softraggedright

\bibitem{honeynet}
The Honeynet Project. \url{http://www.honeynet.org/}
\bibitem{honeynet-15}
The Honeynet Project \emph{Scan of the Month} \#15.
\url{http://old.honeynet.org/scans/scan15/}
\bibitem{borland-honeynet}
Borland, Matt. Submission to Honeynet.org \emph{Scan of the Month},
05/05/2001. \url{http://old.honeynet.org/scans/scan15/som/som6.txt}

\end{thebibliography}


\end{document}
